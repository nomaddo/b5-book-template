% 大体tex-live-fullとか入れたら入る
\documentclass[9pt,b5paper,dvipdfmx,openany]{jsbook}
\usepackage[utf8]{inputenc}
\usepackage[dvipdfmx]{graphicx}
\usepackage{ascmac} % 枠付き環境のため
\usepackage{amsmath}
\usepackage[dvipdfmx]{hyperref}
\usepackage{pxjahyper}
\usepackage{xcolor}
\usepackage{listings,jlisting} % jlistingはtex-live-fullでも入ってない。手動
\usepackage{fancyvrb}
\usepackage{wrapfig}


% \usepackage{courier}
%\usepackage{DejaVuSansMono}
\usepackage[T1]{fontenc}
% \usepackage{lmodern}
% \usepackage{luximono}
% \usepackage[scaled=0.85]{beramono}

% \date{\today}

\setlength{\textwidth}{\fullwidth}
\setlength{\evensidemargin}{\oddsidemargin}

% 適当に編集する、まぁコレに合わせてやれば良いんじゃないかな
\lstset{
  language={C},% プログラミング言語
  basicstyle={\ttfamily\small},% ソースコードのテキストのスタイル
  keywordstyle={\bfseries},% 予約語等のキーワードのスタイル
  commentstyle={},% コメントのスタイル
  stringstyle={},% 文字列のスタイル
  frame=trlb,% ソースコードの枠線の設定 (none だと非表示)
  numbers=none,% 行番号の表示 (left だと左に表示)
  numberstyle={},% 行番号のスタイル
  xleftmargin=5pt,% 左余白
  xrightmargin=5pt,% 右余白
  keepspaces=true,% 空白を表示する
  showspaces=false,
  showstringspaces=false,
  mathescape=false,% $ で囲った部分を数式として表示する ($ がソースコード中で使えなくなるので注意)
  backgroundcolor=\color[gray]{0.8},
  % 手動強調表示の設定
  moredelim=[is][\fontfamily{pcr}\selectfont\itshape]{@/}{/@},
  moredelim=[is][\color{red}]{@r\{}{\}@},
  moredelim=[is][\color{blue}]{@b\{}{\}@},
  moredelim=[is][\color{DarkGreen}]{@g\{}{\}@},
% moredelim=[is][\fontfamily{pcr}\itshape]{@@}{@},
}

\newcommand{\prog}[1]{\colorbox[gray]{0.8}{\lstinline{#1}}}

\title{素敵な本のタイトル}
\author{偉大な本書きの名前}

\begin{document}

\maketitle
\tableofcontents

\chapter{ちゃぷたー1}
\begin{flushright}
  \it カッコいい系謎の名言を書く - 作者不明 \\
\end{flushright}

\section{ほげー}

素敵な文章。
\footnote{素敵な注釈。あんまり注釈だらけにならないようにね。}

\begin{itemize}
\item アイテマイズ1
\item アイテマイズ2
\item アイテマイズ3
\end{itemize}

\subsection{プログラム}
\begin{lstlisting}
// lstlistingはページ跨いで表示されるんで注意な
// 最終的に自分で手動で位置調整する必要が有ると思う
real a;
int b = 1 + 3;
int[3] c;

int f (int i) {
  return i + i;
}
\end{lstlisting}

\begin{minipage}[t]{0.5\linewidth}
\begin{lstlisting}[caption=プログラム]
int f (int a, int b) {
  return (int)((double) a + b);
}
\end{lstlisting}
\end{minipage}
\begin{minipage}[t]{0.5\linewidth}
  \centering
  かえるぴょこぴょこ みぴょこぴょこ \\
  あわせてぴょこぴょこ みぴょこぴょこ
\end{minipage}

\begin{itembox}[l]{あとがき}
  あとのがき \\
  残業は悪い文化
\end{itembox}

\begin{thebibliography}{9}
  \bibitem{ocaml} OCaml \\
    \href{http://ocaml.org}{http://ocaml.org} \\
    大切なことはみんなOCamlとEmacsとクソラノベアニメから学んだんだ。
\end{thebibliography}

\thispagestyle{empty}
\vspace*{\stretch{1}}
\begin{flushright}
  \begin{minipage}{0.5\hsize}
    \begin{description}
    \item{著者:}素敵な名前
    \item{発行:}20XX年YY月ZZ日
    \item{印刷:}素敵な印刷所
    \end{description}
  \end{minipage}
\end{flushright}

\end{document}

%%% Local Variables:
%%% mode: latex
%%% TeX-master: t
%%% End:
